\documentclass[paper=a4, fontsize=11pt]{scrartcl}
%\usepackage{sectsty} % Allows for custom section title styling
\usepackage[T1]{fontenc}
\usepackage{fourier} % We're using the Utopia font because it's great.
\usepackage{enumerate} % Allows for custom enumeration types
\usepackage{tikz} % Graphics powered by TikZ!
\usepackage{xcolor} % More colors
\usepackage{listings} % For shell output
\usepackage{nameref} % Reference sections by name, since we're avoiding section numbering

\usetikzlibrary{positioning} % Relative positioning of nodes
\usetikzlibrary{automata} % Handy stuff for state machine diagrams

\tikzstyle{every state}=[fill=orange!30, draw=orange, very thick]
\tikzstyle{lock}=[->, draw=cyan, very thick, every to/.style={bend left=20}]
\tikzstyle{unlock}=[->, draw=magenta, very thick, every to/.style={bend left=20}]

\lstdefinestyle{ShellStyle} {
  basicstyle=\small\ttfamily,
  numbers=none,
  frame=tblr,
  columns=fullflexible,
  backgroundcolor=\color{blue!10},
  linewidth=0.9\linewidth,
  xleftmargin=0.1\linewidth
}

%\allsectionsfont{\centering \normalfont\scshape} % Center and semi-cap section titles
\newcommand{\horrule}[1]{\rule{\linewidth}{#1}} % Horizontal rule with weight arg

\title{
  \normalfont \normalsize 
  \textsc{New Mexico Tech} \\ [25pt]
  \horrule{0.5pt} \\[0.4cm]
  \huge CSE 325 --- Lab Project 5 \\ Memory Management \\
  \horrule{2pt} \\[0.5cm]
}

\author{Rob Kelly \& Ian Neal \\ SANIC TEEM}
\date{\normalsize\today}

\begin{document}
\maketitle

%%% INTRODUCTION %%%
The SANIC TEEM Maniac Middle Management Memory Minder (hereafter referred to as \textit{the project}) is 

\begin{description}
  \item[First Fit] ,

  \item[Best Fit] ,

  \item[Worst Fit] , and

  \item[Next Fit]
\end{description}

Most of the codebase of this project was given as part of the lab assignment. We have implemented the functionality of the project in the following files:

\begin{itemize}
  \item 

  \item Build targets for the added files and this README in the \texttt{Makefile}.
\end{itemize}

%%% BUILDING %%%
\section*{Building}
\begin{itemize}
  \item \texttt{make} to build the project normally.

  \item \texttt{make test} to run the provided testing harness.

  \item \texttt{make doc} to build this README. Requires \texttt{pdflatex} and a number of \LaTeX\hspace{0em} packages, all of which are included in the popular \textbf{TeX Live} distribution.

  \item \texttt{make clean} to clean up temporary files, build files, and output.
\end{itemize} 

%%% USAGE %%%
% Taken from memorytests.c:main()
\section*{Usage}
\texttt{mem -test <test> <strategy> OR mem -try <arg1> <arg2> ...}

\begin{itemize}
  \item \texttt{test} is .

  \item \texttt{strategy} is .

  \item \texttt{argn} is .
\end{itemize}


%%% EXAMPLES %%%
\section*{Examples}
\begin{lstlisting}[style=ShellStyle]$
\end{lstlisting}

%%% DESIGN %%%
\section*{Design \& Implementation}
% Be sure and update this as we make progress...
\subsection*{Organization}


%% \begin{figure}[h]
%%   \centering
%%   \begin{tikzpicture}[node distance = 2cm, auto]
%%     \node         (insert)                                     {\texttt{enter\_sched\_queue}};
%%     \node[state]  (production)  [below left = of insert]       {\phantom{-}\textit{production}\phantom{-}};
%%     \node[state]  (consumption) [below right = of insert]      {\textit{consumption}};
%%     \node         (remove)      [below left = of consumption]  {\texttt{leave\_sched\_queue}};
%%     \node         (wait)        [below right = of consumption] {\texttt{wait\_for\_queue}};

%%     \draw[lock]   (production)  to (insert);
%%     \draw[unlock] (insert)      to (consumption);
%%     \draw[lock]   (consumption) to (remove);
%%     \draw[unlock] (remove)      to (production);
%%     \draw[lock]   (consumption) to (wait);
%%     \draw[unlock] (wait)        to (consumption);
%%   \end{tikzpicture}
%%   \caption{Scheduling queue capacity synchronization.}
%%   \label{fig:qsync}
%% \end{figure}

%%% BUGS %%%
\section*{Known Bugs}

\end{document}
